\inputencoding{utf8}
\HeaderA{optimsimplex.new}{Creates a Simplex Object}{optimsimplex.new}
\aliasA{optimsimplex.axes}{optimsimplex.new}{optimsimplex.axes}
\aliasA{optimsimplex.coords}{optimsimplex.new}{optimsimplex.coords}
\aliasA{optimsimplex.oriented}{optimsimplex.new}{optimsimplex.oriented}
\aliasA{optimsimplex.pfeffer}{optimsimplex.new}{optimsimplex.pfeffer}
\aliasA{optimsimplex.randbounds}{optimsimplex.new}{optimsimplex.randbounds}
\aliasA{optimsimplex.spendley}{optimsimplex.new}{optimsimplex.spendley}
\keyword{method}{optimsimplex.new}
%
\begin{Description}\relax
\code{optimsimplex.new} creates a simplex list object which contains, among
other elements, a matrix of vertices and a vector of function values
calculated at those vertices. The object is actually created by a secondary
function based upon the value of the \code{method} argument:\begin{description}

\item[NULL] -> \code{optimsimplex.coords}
\item['axes'] -> \code{optimsimplex.axes}
\item['pfeffer'] -> \code{optimsimplex.pfeffer}
\item['randbounds'] ->  \code{optimsimplex.randbounds}
\item['spendley'] ->  \code{optimsimplex.spendley}
\item['oriented'] -> \code{optimsimplex.oriented}

\end{description}

\end{Description}
%
\begin{Usage}
\begin{verbatim}
  optimsimplex.new(coords = NULL, fun = NULL, data = NULL, method = NULL,
                   x0 = NULL, len = NULL, deltausual = NULL, deltazero = NULL,
                   boundsmax = NULL, boundsmin = NULL, nbve = NULL,
                   simplex0 = NULL)
  optimsimplex.coords(coords = NULL, fun = NULL, data = NULL)
  optimsimplex.axes(x0 = NULL, fun = NULL, len = NULL, data = NULL)
  optimsimplex.pfeffer(x0 = NULL, fun = NULL, deltausual = NULL,
                       deltazero = NULL, data = NULL)
  optimsimplex.randbounds(x0 = NULL, fun = NULL, boundsmin = NULL,
                          boundsmax = NULL, nbve = NULL, data = NULL)
  optimsimplex.spendley(x0 = NULL, fun = NULL, len = NULL, data = NULL)
  optimsimplex.oriented(simplex0 = NULL, fun = NULL, data = NULL)
\end{verbatim}
\end{Usage}
%
\begin{Arguments}
\begin{ldescription}
\item[\code{coords}] The matrix of point estimate coordinates in the simplex. The
coords matrix is expected to be a nbve x n matrix, where n is the dimension
of the space and nbve is the number of vertices in the simplex, with nbve>=
n+1. Only used if \code{method} is set to NULL.
\item[\code{fun}] The function to compute at vertices. The function is expected to
have the following input and output arguments:

\Tabular{l}{
myfunction <- function(x, this)\{ \\{}
... \\{}
return(list(f=f,this=this)) \\{}
\}
}
where x is a row vector and this a user-defined data, i.e. the \code{data}
argument.
\item[\code{data}] A user-defined data passed to the function. If data is provided,
it is passed to the callback function both as an input and output argument.
\code{data} may be used if the function uses some additionnal parameters. It
is returned as an output parameter because the function may modify the data
while computing the function value. This feature may be used, for example,
to count the number of times that the function has been called.
\item[\code{method}] The method used to create the new optimsimplex object, either
'axes', 'pfeffer', 'randbounds', 'spendley' or 'oriented'.
\item[\code{x0}] The initial point estimates, as a row vector of length n.
\item[\code{len}] The dimension of the simplex. If length is a value, that unique
length is used in all directions. If length is a vector with n values, each
length is used with the corresponding direction. Only used if \code{method}
is set to 'axes' or 'spendley'.
\item[\code{deltausual}] The absolute delta for non-zero values. Only used if
\code{method} is set to 'pfeffer'.
\item[\code{deltazero}] The absolute delta for zero values. Only used if
\code{method} is set to 'pfeffer'.
\item[\code{boundsmin}] A vector of minimum bounds. Only used if \code{method} is
set to 'randbounds'.
\item[\code{boundsmax}] A vector of maximum bounds. Only used if \code{method} is
set to 'randbounds'.
\item[\code{nbve}] The total number of vertices in the simplex. Only used if
\code{method} is set to 'randbounds'.
\item[\code{simplex0}] The initial simplex. Only used if \code{method} is set to
'oriented'.
\end{ldescription}
\end{Arguments}
%
\begin{Details}\relax
All arguments of \code{optimsimplex.new} are optional. If no input is provided,
the new simplex object is empty.

If \code{method} is NULL, the new simplex object is created by
\code{optimsimplex.coords}. If \code{coords} is NULL, the simplex object is
empty; otherwise, \code{coords} is used as the initial vertice coordinates
in the new simplex.

If \code{method} is set to 'axes', the new simplex object is created by
\code{optimsimplex.axes}. The initial vertice coordinates are stored in a nbve
x n matrix built as follows:
\Tabular{llcccrclcccr}{
[,1] &| & x0[1] &.... & x0[n] & | & & | & len[1]
&... & 0 &| \\{}
[,.] &| &... &... &... & | & + & | &... &...
&... &| \\{}
[,nbve] &| & x0[1] &... & x0[n] & | & & | & 0
&... & len[n] &| \\{}
}

If \code{method} is set to 'pfeffer', the new simplex object is created by
\code{optimsimplex.pfeffer} using the Pfeffer's method, i.e. a relative delta
for non-zero values and an absolute delta for zero values.

If \code{method} is set to 'randbounds', the new simplex object is created by
\code{optimsimplex.randbounds}. The initial vertice coordinates are stored in
a nbve x n matrix consisting of the initial point estimates (on the first
row) and a (nbve-1) x n matrix of randomly sampled numbers between the
specified the bounds. The number of vertices \code{nbve} in the simplex is
arbitrary.

If \code{method} is set to 'spendley', the new simplex object is created by
\code{optimsimplex.spendley} using the Spendely's method, i.e. a regular
simplex made of nbve = n+1 vertices. 

If \code{method} is set to 'oriented', the new simplex object is created by
\code{optimsimplex.oriented} in sorted order. The new simplex has the same
sigma- length of the base simplex, but is "oriented" depending on the function
value. The created simplex may be used, as Kelley suggests, for a restart of
Nelder-Mead algorithm.
\end{Details}
%
\begin{Value}
Return a list with the following elements: \begin{description}

\item[newobj] A list with a 'type' attribute set to 'T\_SIMPLEX' and with
the following elements: \begin{description}

\item[verbose] The verbose option, controlling the amount of messages.
Set to 0.
\item[x] The coordinates of the vertices, with size nbve x n.
\item[n] The dimension of the space.
\item[fv] The values of the function at given vertices. It is a column
matrix of length nbve.
\item[nbve] The number of vertices.

\end{description}


\item[data] The updated \code{data} input argument.

\end{description}

\end{Value}
%
\begin{Author}\relax
Author of Scilab optimsimplex module: Michael Baudin (INRIA - Digiteo)

Author of R adaptation: Sebastien Bihorel (\email{sb.pmlab@gmail.com})
\end{Author}
%
\begin{References}\relax
"A Simplex Method for Function Minimization", Nelder, J. A. and Mead, R. The
Computer Journal, January, 1965, 308-313

"Sequential Application of Simplex Designs in Optimisation and Evolutionary
Operation", W. Spendley, G. R. Hext, F. R. Himsworth, Technometrics, Vol. 4, No.
4 (Nov., 1962), pp. 441-461, Section 3.1

"A New Method of Constrained Optimization and a Comparison With Other Methods",
M. J. Box, The Computer Journal 1965 8(1):42-52, 1965 by British Computer
Society

"Detection and Remediation of Stagnation in the Nelder-Mead Algorithm Using a
Sufficient Decrease Condition", SIAM J. on Optimization, Kelley C.T., 1999

"Multi-Directional Search: A Direct Search Algorithm for Parallel Machines", by
E. Boyd, Kenneth W. Kennedy, Richard A. Tapia, Virginia Joanne Torczon,
Virginia Joanne Torczon, 1989, Phd Thesis, Rice University

"Grid Restrained Nelder-Mead Algorithm", Arpad Burmen, Janez Puhan, Tadej Tuma,
Computational Optimization and Applications, Volume 34 , Issue 3 (July 2006),
Pages: 359 - 375

"A convergent variant of the Nelder-Mead algorithm", C. J. Price, I. D. Coope,
D. Byatt, Journal of Optimization Theory and Applications, Volume 113 , Issue 1
(April 2002), Pages: 5 - 19,

"Global Optimization Of Lennard-Jones Atomic Clusters", Ellen Fan, Thesis,
February 26, 2002, McMaster University
\end{References}
%
\begin{Examples}
\begin{ExampleCode}
  myfun <- function(x,this){return(list(f=sum(x^2),this=this))}
  mat <- matrix(c(0,1,0,0,0,1),ncol=2)
  
  optimsimplex.new()
  optimsimplex.new(coords=mat,x0=1:4,fun=myfun)
  optimsimplex.new(method='axes',x0=1:4,fun=myfun)
  optimsimplex.new(method='pfeffer',x0=1:6,fun=myfun)
  opt <- optimsimplex.new(method='randbounds',x0=1:6,boundsmin=rep(0,6),
                          boundsmax=rep(10,6),fun=myfun)
  opt
  optimsimplex.new(method='spendley',x0=1:6,fun=myfun,len=10)
  optimsimplex.new(method='oriented',simplex=opt$newobj,fun=myfun)
 
\end{ExampleCode}
\end{Examples}

\inputencoding{utf8}
\HeaderA{optimsimplex.print}{Simplex Formatting and Display}{optimsimplex.print}
\aliasA{optimsimplex.tostring}{optimsimplex.print}{optimsimplex.tostring}
\keyword{method}{optimsimplex.print}
%
\begin{Description}\relax
\code{optimsimplex.tostring} formats the coordinates and function values in a
character vector.

\code{optimsimplex.print} displays to screen the content of the current
simplex with dimensions, coordinates and function values. This function calls
\code{optimsimplex.tostring} to format the content of the simplex.
\end{Description}
%
\begin{Usage}
\begin{verbatim}
  optimsimplex.print(this = NULL)
  optimsimplex.tostring(this = NULL)
\end{verbatim}
\end{Usage}
%
\begin{Arguments}
\begin{ldescription}
\item[\code{this}] A simplex object.
\end{ldescription}
\end{Arguments}
%
\begin{Value}
\code{optimsimplex.tostring} returns a vector of character string of length
nbve, where nbve is the number of vertices.

\code{optimsimplex.print} does not return any value but print to screen (or
log file) the content of the current simplex.
\end{Value}
%
\begin{Author}\relax
Author of Scilab optimsimplex module: Michael Baudin (INRIA - Digiteo)

Author of R adaptation: Sebastien Bihorel (\email{sb.pmlab@gmail.com})
\end{Author}
%
\begin{SeeAlso}\relax
\code{\LinkA{optimsimplex.new}{optimsimplex.new}}
\end{SeeAlso}
%
\begin{Examples}
\begin{ExampleCode}
  opt <- optimsimplex.new(method='axes',x0=1:5)$newobj
  optimsimplex.tostring(opt)
  optimsimplex.print(opt)
\end{ExampleCode}
\end{Examples}

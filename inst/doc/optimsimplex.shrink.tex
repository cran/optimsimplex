\inputencoding{utf8}
\HeaderA{optimsimplex.shrink}{Simplex Shrink}{optimsimplex.shrink}
\keyword{method}{optimsimplex.shrink}
%
\begin{Description}\relax
This function shrinks the simplex with given coefficient sigma and returns an
updated simplex. The shrink is performed with respect to the first point in
the simplex.
\end{Description}
%
\begin{Usage}
\begin{verbatim}
  optimsimplex.shrink(this = NULL, fun = NULL, sigma = 0.5, data = NULL)
\end{verbatim}
\end{Usage}
%
\begin{Arguments}
\begin{ldescription}
\item[\code{this}] An simplex object
\item[\code{fun}] The function to compute at vertices. The function is expected to
have the following input and output arguments:

\Tabular{l}{
myfunction <- function(x, this)\{ \\{}
... \\{}
return(list(f=f,this=this)) \\{}
\}
}
where x is a row vector and this a user-defined data, i.e. the \code{data}.
\item[\code{sigma}] The shrinkage coefficient. The default value is 0.5.
\item[\code{data}] A user-defined data passed to the function. If data is provided,
it is passed to the callback function both as an input and output argument.
\code{data} may be used if the function uses some additionnal parameters. It
is returned as an output parameter because the function may modify the data
while computing the function value. This feature may be used, for example,
to count the number of times that the function has been called.
\end{ldescription}
\end{Arguments}
%
\begin{Value}
Return a list with the following elements: \begin{description}

\item[this] The updated simplex object.
\item[data] The updated user-defined data.

\end{description}

\end{Value}
%
\begin{Author}\relax
Author of Scilab optimsimplex module: Michael Baudin (INRIA - Digiteo)

Author of R adaptation: Sebastien Bihorel (\email{sb.pmlab@gmail.com})
\end{Author}
%
\begin{SeeAlso}\relax
\code{\LinkA{optimsimplex.new}{optimsimplex.new}}
\end{SeeAlso}

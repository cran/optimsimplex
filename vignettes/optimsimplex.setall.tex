\inputencoding{utf8}
\HeaderA{Set functions}{Optimsimplex Set Function Class}{Set functions}
\aliasA{optimsimplex.setall}{Set functions}{optimsimplex.setall}
\aliasA{optimsimplex.setallfv}{Set functions}{optimsimplex.setallfv}
\aliasA{optimsimplex.setallx}{Set functions}{optimsimplex.setallx}
\aliasA{optimsimplex.setfv}{Set functions}{optimsimplex.setfv}
\aliasA{optimsimplex.setn}{Set functions}{optimsimplex.setn}
\aliasA{optimsimplex.setnbve}{Set functions}{optimsimplex.setnbve}
\aliasA{optimsimplex.setve}{Set functions}{optimsimplex.setve}
\aliasA{optimsimplex.setx}{Set functions}{optimsimplex.setx}
\keyword{method}{Set functions}
%
\begin{Description}\relax
The functions assign content to various elements of a simplex object:
\begin{description}

\item[\code{optimsimplex.setall}] Set all the coordinates and the function
values of all the vertices.
\item[\code{optimsimplex.setallfv}] Set all the function values of all the
vertices.
\item[\code{optimsimplex.setallx}] Set all the coordinates of all the
vertices.
\item[\code{optimsimplex.setfv}] Set the function value at a givenindex.
\item[\code{optimsimplex.setn}] Set the dimension of the space of the
simplex.
\item[\code{optimsimplex.setnbve}] Set the number of vertices of the
simplex.
\item[\code{optimsimplex.setve}] Set the coordinates of the vertex and the
function values at a given index in the current simplex.
\item[\code{optimsimplex.setx}] Set the coordinates of the vertex at a given
index in the current simplex.

\end{description}

\end{Description}
%
\begin{Usage}
\begin{verbatim}
  optimsimplex.setall(this = NULL, simplex = NULL)
  optimsimplex.setallfv(this = NULL, fv = NULL)
  optimsimplex.setallx(this = NULL, x = NULL)
  optimsimplex.setfv(this = NULL, ive = NULL, fv = NULL)
  optimsimplex.setn(this = NULL, n = NULL)
  optimsimplex.setnbve(this = NULL, nbve = NULL)
  optimsimplex.setve(this = NULL, ive = NULL, fv = NULL, x = NULL)
  optimsimplex.setx(this = NULL, ive = NULL, x = NULL)
\end{verbatim}
\end{Usage}
%
\begin{Arguments}
\begin{ldescription}
\item[\code{this}] A simplex object.
\item[\code{simplex}] The simplex to set. It is expected to be a nbve x n+1 matrix
where n is the dimension of the space, nbve is the number of vertices and
with the following content: \begin{itemize}

\item \code{simplex[k,1]} is the function value of the vertex k, with k =
1 to nbve,
\item \code{simplex[k,2:(n+1)]} is the coordinates of the vertex k, with
k = 1 to nbve.

\end{itemize}


\item[\code{fv}] A row vector of function values; \code{fv[k]} is expected to be the
function value for the vertex k, with k = 1 to nbve. For
\code{optimsimplex.setfv}, \code{fv} is expected to be a numerical scalar.
\item[\code{x}] The nbve x n matrix of vertice coordinates; the vertex is expected to
be stored in \code{x[k,1:n]}, with k = 1 to nbve. For
\code{optimsimplex.setve} and \code{optimsimplex.setx}, \code{x} is expected
to be a row matrix.
\item[\code{ive}] Vertex index.
\item[\code{n}] The dimension of the space of the simplex.
\item[\code{nbve}] The number of vertices of the simplex.
\end{ldescription}
\end{Arguments}
%
\begin{Value}
Return a updated simplex object \code{this}.
\end{Value}
%
\begin{Author}\relax
Author of Scilab optimsimplex module: Michael Baudin (INRIA - Digiteo)

Author of R adaptation: Sebastien Bihorel (\email{sb.pmlab@gmail.com})
\end{Author}
%
\begin{SeeAlso}\relax
\code{\LinkA{optimsimplex}{optimsimplex}}
\end{SeeAlso}

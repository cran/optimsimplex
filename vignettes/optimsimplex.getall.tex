\inputencoding{utf8}
\HeaderA{Get functions}{Optimsimplex Get Function Class}{Get functions}
\aliasA{optimsimplex.getall}{Get functions}{optimsimplex.getall}
\aliasA{optimsimplex.getallfv}{Get functions}{optimsimplex.getallfv}
\aliasA{optimsimplex.getallx}{Get functions}{optimsimplex.getallx}
\aliasA{optimsimplex.getfv}{Get functions}{optimsimplex.getfv}
\aliasA{optimsimplex.getn}{Get functions}{optimsimplex.getn}
\aliasA{optimsimplex.getnbve}{Get functions}{optimsimplex.getnbve}
\aliasA{optimsimplex.getve}{Get functions}{optimsimplex.getve}
\aliasA{optimsimplex.getx}{Get functions}{optimsimplex.getx}
\keyword{method}{Get functions}
%
\begin{Description}\relax
The functions extract the content to various elements of a simplex object:
\begin{description}

\item[\code{optimsimplex.getall}] Get all the coordinates and the function
values of all the vertices.
\item[\code{optimsimplex.getallfv}] Get all the function values of all the
vertices.
\item[\code{optimsimplex.getallx}] Get all the coordinates of all the
vertices.
\item[\code{optimsimplex.getfv}] Get the function value at a given index.
\item[\code{optimsimplex.getn}] Get the dimension of the space of the
simplex.
\item[\code{optimsimplex.getnbve}] Get the number of vertices of the
simplex.
\item[\code{optimsimplex.getve}] Get the vertex at a given index in the
current simplex.
\item[\code{optimsimplex.getx}] Get the coordinates of the vertex at a given
index in the current simplex.

\end{description}

\end{Description}
%
\begin{Usage}
\begin{verbatim}
  optimsimplex.getall(this = NULL)
  optimsimplex.getallfv(this = NULL)
  optimsimplex.getallx(this = NULL)
  optimsimplex.getfv(this = NULL, ive = NULL)
  optimsimplex.getn(this = NULL)
  optimsimplex.getnbve(this = NULL)
  optimsimplex.getve(this = NULL, ive = NULL)
  optimsimplex.getx(this = NULL, ive = NULL)
\end{verbatim}
\end{Usage}
%
\begin{Arguments}
\begin{ldescription}
\item[\code{this}] A simplex object.
\item[\code{ive}] Vertex index.
\end{ldescription}
\end{Arguments}
%
\begin{Value}
\begin{description}

\item[\code{optimsimplex.getall}] Return a nbve x n+1 matrix, where n is the
dimension of the space, nbve is the number of vertices and with the
following content: \begin{itemize}

\item \code{simplex[k,1]} is the function value of the vertex k, with k =
1 to nbve,
\item \code{simplex[k,2:(n+1)]} is the coordinates of the vertex k, with
k = 1 to nbve.

\end{itemize}


\item[\code{optimsimplex.getallfv}] Return a row vector of function values,
which k\textasciicircum{}th element is the function value for the vertex k, with k = 1 to
nbve.
\item[\code{optimsimplex.getallx}] Return a nbve x n matrix of vertice
coordinates; any given vertex is expected to be stored at row k, with k =
1 to nbve.
\item[\code{optimsimplex.getfv}] Return a numeric scalar.
\item[\code{optimsimplex.getn}] Return a numeric scalar.
\item[\code{optimsimplex.getnbve}] Return a numeric scalar.
\item[\code{optimsimplex.getve}] Return an object of class 'vertex', i.e. a
list with the following elements: \begin{description}

\item[n] The dimension of the space of the simplex.
\item[x] The coordinates of the vertex at index \code{ive}.
\item[fv] The value of the function at index \code{ive}.

\end{description}


\item[optimsimplex.getx] Return a row vector, representing the coordinates
of the vertex at index \code{ive}.


\end{description}

\end{Value}
%
\begin{Author}\relax
Author of Scilab optimsimplex module: Michael Baudin (INRIA - Digiteo)

Author of R adaptation: Sebastien Bihorel (\email{sb.pmlab@gmail.com})
\end{Author}
%
\begin{SeeAlso}\relax
\code{\LinkA{optimsimplex}{optimsimplex}}
\end{SeeAlso}

\inputencoding{utf8}
\HeaderA{osimplex}{S3 osimplex and vertex classes}{osimplex}
\aliasA{is.osimplex}{osimplex}{is.osimplex}
\aliasA{is.vertex}{osimplex}{is.vertex}
\aliasA{print.osimplex}{osimplex}{print.osimplex}
\aliasA{print.vertex}{osimplex}{print.vertex}
\aliasA{vertex}{osimplex}{vertex}
\keyword{method}{osimplex}
%
\begin{Description}\relax
These functions support the S3 classes 'osimplex' and 'vertex'. They are 
intended to either create objects of these classes or check if an object is 
of these classes
\end{Description}
%
\begin{Usage}
\begin{verbatim}
  osimplex(verbose,x,n,fv,nbve)
  
  vertex(x,n,fv)
  
  ## S3 method for class 'osimplex'
print(x,...)
  
  ## S3 method for class 'vertex'
print(x,...)
  
  ## S3 method for class 'osimplex'
is(x)
  
  ## S3 method for class 'vertex'
is(x)
\end{verbatim}
\end{Usage}
%
\begin{Arguments}
\begin{ldescription}
\item[\code{verbose}] The verbose option, controlling the amount of messages
\item[\code{x}] The coordinates of the vertices, with size nbve x n in a simplex 
object or 1 x n in a vertex.
\item[\code{n}] The dimension of the space.
\item[\code{fv}] The values of the function at given vertices. It is a column
matrix of length nbve in a simplex or a single value in a vertex.
\item[\code{nbve}] The number of vertices in a simplex.
\item[\code{...}] optional arguments to 'print' or 'plot' methods.
\end{ldescription}
\end{Arguments}
%
\begin{Details}\relax
A simplex of size n x nbve is essentially a collection of vertex of size n. 
\end{Details}
%
\begin{Value}
\code{osimplex} returns a list with the following elements: verbose, x, n, fv, 
and nbve.
\code{vertex} returns a list with the following elements: x, n, and fv.
\end{Value}
%
\begin{Author}\relax
Author of Scilab optimsimplex module: Michael Baudin (INRIA - Digiteo)

Author of R adaptation: Sebastien Bihorel (\email{sb.pmlab@gmail.com})
\end{Author}

\inputencoding{utf8}
\HeaderA{optimsimplex.utils}{Optimsimplex Utility Functions}{optimsimplex.utils}
\aliasA{optimsimplex.center}{optimsimplex.utils}{optimsimplex.center}
\aliasA{optimsimplex.check}{optimsimplex.utils}{optimsimplex.check}
\aliasA{optimsimplex.deltafv}{optimsimplex.utils}{optimsimplex.deltafv}
\aliasA{optimsimplex.deltafvmax}{optimsimplex.utils}{optimsimplex.deltafvmax}
\aliasA{optimsimplex.dirmat}{optimsimplex.utils}{optimsimplex.dirmat}
\aliasA{optimsimplex.fvmean}{optimsimplex.utils}{optimsimplex.fvmean}
\aliasA{optimsimplex.fvstdev}{optimsimplex.utils}{optimsimplex.fvstdev}
\aliasA{optimsimplex.fvvariance}{optimsimplex.utils}{optimsimplex.fvvariance}
\aliasA{optimsimplex.size}{optimsimplex.utils}{optimsimplex.size}
\aliasA{optimsimplex.sort}{optimsimplex.utils}{optimsimplex.sort}
\aliasA{optimsimplex.xbar}{optimsimplex.utils}{optimsimplex.xbar}
\keyword{method}{optimsimplex.utils}
%
\begin{Description}\relax
These functions enable various calculations and checks on the current simplex:
\begin{description}

\item[\code{optimsimplex.center}] Compute the center of the current
simplex.
\item[\code{optimsimplex.check}] Check the consistency of the data in the
current simplex.
\item[\code{optimsimplex.deltafv}] Compute the vector of function value
differences with respect to the function value at the first vertex (the
lowest).
\item[\code{optimsimplex.deltafvmax}] Compute the difference of function
value between the lowest and the highest vertices. It is expected that the
first vertex (\code{this\$x[1,]}) is associated with the smallest function
value and that the last vertex (\code{this\$x[nbve,]}) is associated with
the highest function value.
\item[\code{optimsimplex.dirmat}] Compute the matrix of simplex direction,
i.e. the matrix of differences of vertice coordinates with respect to the
first vertex.
\item[\code{optimsimplex.fvmean}] Compute the mean of the function values in
the current simplex.
\item[\code{optimsimplex.fvstdev}] Compute the standard deviation of the
function values in the current simplex.
\item[\code{optimsimplex.fvvariance}] Compute the variance of the function
values in the current simplex.
\item[\code{optimsimplex.size}] Determines the size of the simplex.
\item[\code{optimsimplex.sort}] Sort the simplex by increasing order of
function value, so the smallest function is at the first vertex.
\item[\code{optimsimplex.xbar}] Compute the center of n vertices, by
excluding the vertex with index \code{iexcl}. The default of \code{iexcl}
is the number of vertices: in that case, if the simplex is sorted in
increasing function value order, the worst vertex is excluded.

\end{description}

\end{Description}
%
\begin{Usage}
\begin{verbatim}
  optimsimplex.center(this = NULL)
  optimsimplex.check(this = NULL)
  optimsimplex.deltafv(this = NULL)
  optimsimplex.deltafvmax(this = NULL)
  optimsimplex.dirmat(this = NULL)
  optimsimplex.fvmean(this = NULL)
  optimsimplex.fvstdev(this = NULL)
  optimsimplex.fvvariance(this = NULL)
  optimsimplex.size(this = NULL, method = NULL)
  optimsimplex.sort(this = NULL)
  optimsimplex.xbar(this = NULL, iexcl = NULL)
\end{verbatim}
\end{Usage}
%
\begin{Arguments}
\begin{ldescription}
\item[\code{this}] The current simplex.
\item[\code{method}] The method to use to compute the size of the simplex. The
available methods are the following: \begin{description}

\item['sigmaplus'] (this is the default) The sigmamplus size is the
maximum 2-norm length of the vector from each vertex to the first
vertex. It requires one loop over the vertices.
\item['sigmaminus'] The sigmaminus size is the minimum 2-norm length of
the vector from each vertex to the first vertex. It requires one loop
over the vertices.
\item['Nash'] The 'Nash' size is the sum of the norm of the norm-1 length
of the vector from the given vertex to the first vertex. It requires one
loop over the vertices.
\item['diameter'] The diameter is the maximum norm-2 length of all the
edges of the simplex. It requires 2 nested loops over the vertices.

\end{description}


\item[\code{iexcl}] The index of the vertex to exclude in center computation.
\end{ldescription}
\end{Arguments}
%
\begin{Value}
\begin{description}

\item[\code{optimsimplex.center}] Return a vector of length nbve, where nbve
is the number of vertices in the current simplex.
\item[\code{optimsimplex.check}] Return an error message if the dimensions
of the various elements of the current simplex do not match.
\item[\code{optimsimplex.deltafv}] Return a column vector of length nbve-1.
\item[\code{optimsimplex.deltafvmax}] Return a numeric scalar.
\item[\code{optimsimplex.dirmat}] Return a n x n numeric matrix, where n is
the dimension of the space of the simplex.
\item[\code{optimsimplex.fvmean}] Return a numeric scalar.
\item[\code{optimsimplex.fvstdev}] Return a numeric scalar.
\item[\code{optimsimplex.fvvariance}] Return a numeric scalar.
\item[\code{optimsimplex.size}] Return a numeric scalar.
\item[\code{optimsimplex.sort}] Return an updated simplex object.
\item[\code{optimsimplex.xbar}] Return a row vector of length n.

\end{description}

\end{Value}
%
\begin{Author}\relax
Author of Scilab optimsimplex module: Michael Baudin (INRIA - Digiteo)

Author of R adaptation: Sebastien Bihorel (\email{sb.pmlab@gmail.com})
\end{Author}
%
\begin{References}\relax
"Compact Numerical Methods For Computers - Linear Algebra and Function
Minimization", J.C. Nash, 1990, Chapter 14. Direct Search Methods

"Iterative Methods for Optimization", C.T. Kelley, 1999, Chapter 6., section
6.2
\end{References}
%
\begin{SeeAlso}\relax
\code{\LinkA{optimsimplex}{optimsimplex}}
\end{SeeAlso}
